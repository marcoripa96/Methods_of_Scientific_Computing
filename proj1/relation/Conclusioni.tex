\section{Conclusioni}\label{conclusioni}
Le impressioni maturate durante questo lavoro in termini di usabilità favoriscono Matlab perché, come già detto in precedenza, è stata trovata una community più ampia e attiva e una buona documentazione, permettendo uno sviluppo senza particolari difficoltà. Al contrario Octave, pur mantenendo un'alta usabilità, è risultato meno completo non offrendo alcune funzionalità presenti in Matlab. Sia su Linux che su Windows l'usabilità dei due softwares resta la medesima.\\\\
Per quanto riguarda le performance si è complessivamente osservata una superiorità da parte di Matlab, sebbene Octave resti un'alternativa più che valida. Dai risultati ottenuti si può inoltre dire che la combinazione migliore risulta essere quella di Windows-Matlab, mentre la peggiore quella di Linux-Octave. In ogni caso, ognuna delle combinazioni analizzate permette di arrivare alla risoluzione di sistemi lineari dopo la decomposizione di Cholesky senza troppe difficoltà. Essendo i risultati ottenuti generalmente buoni, fare una scelta operativa dipende quindi strettamente dalle politiche aziendali. \\\\
In conclusione, avendo la possibilità di scegliere senza vincoli economici, sicuramente la scelta migliore risulta essere quella di Matlab a prescindere dal sistema operativo che si intende utilizzare. Se invece fosse necessario l'utilizzo di Octave sarebbe preferibile il suo utilizzo su una macchina Windows. 