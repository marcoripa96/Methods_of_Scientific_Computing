\subsection{Struttura del codice}
Il codice sorgente di questa parte del progetto è disponibile all'indirizzo \url{https://gitlab.com/m.ripamonti/mcs_projects/-/tree/master/proj2/python/part2/src}. \\
Per essere eseguito è necessaria l'installazione di python 3 e dei requirements definiti nel file \emph{requirements.txt} reperibile al seguente link: \url{https://gitlab.com/m.ripamonti/mcs_projects/-/blob/master/proj2/python/requirements.txt}.
La struttura di organizzazione del codice è la seguente:
\begin{itemize}
    \item \textbf{assets}
    \item \textbf{components}
    \begin{itemize}
        \item content.py
        \item main\_window.py
        \item toolbar.py
    \end{itemize}
    \item \textbf{shared}
    \begin{itemize}
        \item imageCompressor.py
        \item worker.py
    \end{itemize}
    \item main.py
\end{itemize}
La cartella \emph{assets} ha lo scopo di raccogliere file (es: logo, font) che verranno utilizzati dalle componenti grafiche. \emph{components} contiene i file dell'implementazione dell'interfaccia grafica. \emph{shared}, invece, contiene la logica dell'algoritmo di compressione e una classe per costruire un oggetto thread utilizzata per migliorare l'interfaccia grafica. In particolare, quest'ultima cartella ha lo scopo di raccogliere tutti i file le cui funzioni possono essere riutilizzate all'interno del progetto anche in più componenti. Infine, allo stesso livello delle cartelle si trova il file \emph{main.py} che rende possibile l'avvio dell'applicazione.\\
% requirements. non parte niente senza i requiements
I dettagli di quanto implementato saranno discussi nelle apposite sezioni \ref{algoritmo} e \ref{interfaccia}.