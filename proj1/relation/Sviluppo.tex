\section{Sviluppo}\label{sviluppo}
In questa sezione verrà mostrato il procedimento adottato per la fase sperimentale di questo lavoro. \\
Verranno prima descritte brevemente le principali funzioni implementate, poi illustrate le varie fasi dello sviluppo che hanno portato alla realizzazione del progetto. \\
In particolare il lavoro è stato svolto e completato a partire da Matlab, dopodiché si è passati allo studio di Octave con cui si è verificata la riproducibilità di quanto fatto precedentemente.
Dato che per entrambi gli ambienti si è cercato di mantenere la medesima struttura del codice, le funzioni e le fasi presentate in seguito possono essere generalizzate rispetto ai due softwares.

\subsection{Funzioni implementate}
Il codice sorgente di questo progetto è disponibile all'indirizzo \url{https://gitlab.com/m.ripamonti/mcs_projects/-/tree/master/proj1/src} .\\
Le funzioni principali su cui si basa il lavoro sono le seguenti:
\begin{itemize}
    \item~[A, x, b] = readMatrix(filename, usesymamd): prende in input il nome \emph{filename} 
    del file della matrice e un flag \emph{usesymamd} che specifica se effettuare una permutazione 
    sulla matrice; viene restituita in \emph{A} la matrice e, per comodità, in \emph{x} il vettore di 1 delle 
    incognite e in \emph{b} il vettore del termine noto (calcolato a parte dalle \emph{x}).
    \item~[err] = relError(xEs, xApp): prende in input la soluzione esatta \emph{xEs} e la soluzione approssimata \emph{xApp} e restituisce 
    in \emph{err} l'errore relativo calcolato.
    \item~[xApp] = solveWithCholesky(A, b): prende in input una matrice \emph{A} e un vettore dei termini noti \emph{b}, effettua la decomposizione 
    di Cholesky e risolve il sistema; restituisce in \emph{xApp} la soluzione approssimata.
    \item~[err, time] =  matrixAnalyzer(filename, usesymamd): prende in input il nome \emph{filename} del file della matrice e il flag 
    \emph{usesymamd} che specifica se effettuare una permutazione sulla matrice e, appoggiandosi ai metodi precedentemente illustrati, 
    legge la matrice, risolve il sistema e calcola l'errore relativo; vengono restituiti in \emph{err} l'errore relativo e in \emph{time} il tempo impiegato nella risoluzione del sistema.
    \item singlePlotter(table, filterField, filterValue, compareField, compareValue): prende in input 
    una tabella \emph{table} nel formato csv\footnote{\url{https://gitlab.com/m.ripamonti/mcs_projects/-/blob/master/proj1/reports.csv}}
    che viene filtrata in base ai campi \emph{filterField} e \emph{compareField} 
    con i rispettivi valori \emph{filterValue} e \emph{compareValue} e disegna un grafico sulla base dei campi scelti; nel grafico 
    l'asse delle X rappresenta le dimensioni della matrice e l'asse Y, in scala logaritmica, rappresenta tempo, errore e memoria.
    \item comparePlotter(table, filterField, filterValue, compareField, compareValue1, compareValue2): 
    prende in input una tabella \emph{table} nel formato csv che viene filtrata in base ai campi \emph{filterField} e \emph{compareField} con i rispettivi valori \emph{filterValue}, \emph{compareValue1}, 
    \emph{compareValue2} e disegna un grafico sulla base dei campi scelti per il confronto; 
    nel grafico l'asse delle X rappresenta le dimensioni della matrice e l'asse Y, in scala logaritmica, rappresenta tempo, errore e memoria.
    \item~memoryPlotter(matrixName): prende in input il nome \emph{matrixName} di una matrice e traccia un grafico della variazione di 
    memoria utilizzata durante la risoluzione del sistema in funzione del tempo.
    \item~[memory] = memoryReadDelta(filename): prende in input il nome di un file \emph{filename} da cui legge i dati sull'utilizzo di memoria 
    durante la risoluzione del sistema e ne calcola l'escursione togliendo la memoria base usata dal processo; i nuovi valori della memoria vengono 
    restituiti in \emph{memory}
    
\end{itemize}

Le funzioni implementate sono state documentate ed è possibile consultarle tramite il comando \emph{help} sia in Matlab che in Octave.

\subsection{Fase 1: Lettura matrici e permutazione}\label{letturaPermutazione}
Per prima cosa è stata implementata la lettura delle matrici attraverso il metodo \emph{readMatrix}. La lettura avviene mantenendo la 
natura sparsa delle matrici, le quali vengono caricate nel workspace da file \emph{.mat}. Dato lo scopo del progetto, per comodità sono 
stati direttamente integrati nel metodo di lettura anche la generazione dei vettori delle incognite e del termine noto che saranno poi 
utilizzati per la risoluzione del sistema e per il calcolo dell'errore relativo. \\
Come è stato anticipato durante la descrizione delle funzioni, è stato necessario introdurre la possibilità di richiedere una permutazione sulle 
matrici dopo la loro lettura. Questo è dovuto al fatto che il metodo di Cholesky nell'effettuare la decomposizione di alcune matrici sparse 
potrebbe incorrere nel problema del fill-in. Per ovviare a questo problema, esistono diverse strategie di ordinamento che permettono di ridurre 
il numero di elementi diversi da zero derivanti dalla decomposizione. Tra le possibilità di permutazione è stata scelta quella basata sull'algoritmo di 
\emph{approximate minimum degree (amd)}\footnote{\url{https://it.mathworks.com/help/matlab/ref/amd.html}}\footnote{\url{https://octave.sourceforge.io/octave/function/amd.html}}. 
In particolare, dato che tutte le matrici da analizzare sono simmetriche e definite positive, è stata utilizzata l'apposita versione \emph{symamd}\footnote{\url{https://it.mathworks.com/help/matlab/ref/symamd.html}}\footnote{\url{https://octave.sourceforge.io/octave/function/symamd.html}}. 
Come si vedrà nella sezione \ref{symamdVSnotsymamd}, ciò porterà a notevoli benefici nel calcolo della decomposizione di Cholesky.

\subsection{Fase 2: Procedura di risoluzione di un sistema}
Per effettuare la risoluzione di un sistema sfruttando la decomposizione di Cholesky è stato implementato il metodo \emph{solveWithCholesky}. 
Per prima cosa viene effettuata la decomposizione tramite il metodo \emph{chol}\footnote{\url{https://it.mathworks.com/help/matlab/ref/chol.html}}\footnote{\url{https://octave.sourceforge.io/octave/function/chol.html}}.
Per entrambi gli ambienti questo metodo è in grado di controllare che la matrice passata sia definita positiva e restituisce di default 
una matrice triangolare superiore. L'unica differenza sta nel fatto che Matlab è in grado di gestire il caso in cui la matrice non sia 
simmetrica, dando la possibilità di restituire un errore. Octave, invece, procede ugualmente considerando la parte triangolare inferiore come trasposta della superiore
o viceversa.
Dopo aver effettuato la decomposizione e aver ottenuto la matrice \emph{R} tale che  soddisfi $A = R^{T}R$, dove A è la matrice originale, si può procedere alla risoluzione del sistema iniziale eseguendo in sequenza i seguenti due passi:
\begin{enumerate}
    \item Risoluzione del sistema $R^{T}y = b$ mediante il metodo di sostituzione in avanti.
    \item Risoluzione del sistema $Rx = y$ mediante il metodo di sostituzione all'indietro e ricavare la soluzione $x$ approssimata.
\end{enumerate}
Entrambi i sistemi vengono risolti mediante l'operatore '\emph{\textbackslash{}}'\footnote{\url{https://it.mathworks.com/help/matlab/ref/mldivide.html\#bt42oms_head}} che riconoscendo rispettivamente le caratteristiche delle due matrici, la prima triangolare inferiore e la seconda triangolare superiore, applica rispettivamente il metodo di sostituzione in avanti e il metodo di sostituzione all'indietro \cite{quarteroni2006scientific}. \\

\subsection{Fase 3: Raccolta dati relativi alle performance}\label{fase3}
Per agevolare la raccolta dei dati sulle performance per ogni matrice, il metodo implementato è quello di \emph{matrixAnalyzer}. 
Questo metodo era stato inizialmente predisposto in Matlab per calcolare errore relativo, tempo impiegato per la risoluzione e picco di memoria. 
In Octave, però, non esiste alcuna funzione che permetta di monitorare l'utilizzo della RAM. È stato quindi necessario intraprendere un'altra 
strada per mantenere coerenza rispetto al metodo di rilevazione di tale misura di performance. La soluzione adottata è stata quella di utilizzare 
degli scripts\footnote{\url{https://gitlab.com/m.ripamonti/mcs_projects/-/tree/master/proj1/scripts}} in modo da tracciare i valori della memoria occupata dai 
processi di Matlab/Octave ad intervalli di tempo regolari durante la risoluzione del sistema. \\
I passaggi con cui sono stati raccolti i dati sono i seguenti per ogni matrice, ambiente e sistema operativo:
\begin{enumerate}
    \item avvio dello script di monitoraggio memoria
    \item esecuzione metodo \emph{matrixAnalyzer}
    \item inserimento risultati ottenuti in un csv\footnote{\url{https://gitlab.com/m.ripamonti/mcs_projects/-/blob/master/proj1/reports.csv}}
\end{enumerate} 


\subsection{Fase 4: Visualizzazione risultati}
Per quanto riguarda la parte di analisi, sono stati implementati i metodi \emph{singlePlotter} e \emph{comparePlotter} per produrre i grafici richiesti
per la relazione. Questi metodi hanno permesso di automatizzare la lettura da csv per disegnare in modo efficace i grafici sulle performance date le varie 
combinazioni software/OS. \\ % TODO
Inoltre, è stato reputato interessante visualizzare dei grafici aggiuntivi contenenti lo storico delle variazioni di memoria nel tempo per ogni matrice 
a seconda delle diverse configurazioni. A questo scopo è stata implementata la funzione \emph{memoryPlotter}. \\
I grafici prodotti in questa fase saranno discussi in dettaglio alla sezione \ref{graficiMemoria}. 




